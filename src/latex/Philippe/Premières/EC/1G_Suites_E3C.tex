%% LyX 2.3.6.1 created this file.  For more info, see http://www.lyx.org/.
%% Do not edit unless you really know what you are doing.
\documentclass[12pt,french]{article}
\usepackage[T1]{fontenc}
\usepackage[utf8]{inputenc}
\usepackage[a4paper]{geometry}
\geometry{verbose,tmargin=1.5cm,bmargin=1.5cm,lmargin=1.5cm,rmargin=1.5cm,headheight=0.5cm,headsep=0.2cm,footskip=0.5cm}
\usepackage{fancyhdr}
\pagestyle{fancy}
\usepackage{float}
\usepackage{calc}
\usepackage{textcomp}
\usepackage{enumitem}
\usepackage{amsmath}
\usepackage{amsthm}
\usepackage{amssymb}
\usepackage{graphicx}

\makeatletter
%%%%%%%%%%%%%%%%%%%%%%%%%%%%%% Textclass specific LaTeX commands.
\numberwithin{equation}{section}
\numberwithin{figure}{section}
\newlength{\lyxlabelwidth}      % auxiliary length 
\theoremstyle{definition}
    \ifx\thechapter\undefined
      \newtheorem{xca}{\protect\exercisename}
    \else
      \newtheorem{xca}{\protect\exercisename}[chapter]
    \fi
\newenvironment{lyxcode}
	{\par\begin{list}{}{
		\setlength{\rightmargin}{\leftmargin}
		\setlength{\listparindent}{0pt}% needed for AMS classes
		\raggedright
		\setlength{\itemsep}{0pt}
		\setlength{\parsep}{0pt}
		\normalfont\ttfamily}%
	 \item[]}
	{\end{list}}

%%%%%%%%%%%%%%%%%%%%%%%%%%%%%% User specified LaTeX commands.
\usepackage{pst-plot,pst-eucl,pst-math,pstricks-add,pst-xkey,pst-tree}
%\usepackage[np]{numprint}
\usepackage{xcolor,colortbl}
\makeatletter
\define@key[psset]{}{transpalpha}{\pst@checknum{#1}\pstranspalpha}
\psset{transpalpha=1}
\def\psfs@transp{%
  \addto@pscode{/Normal .setblendmode \pstranspalpha .setshapealpha }%
  \psfs@solid}
\makeatother
\newcommand{\kase}[2]{\pspicture[shift=*](#1,#2)\endpspicture}
\newcommand{\dotrule}[1]{\parbox[t]{#1}{\dotfill}}
\renewcommand{\d}{\textrm{d}}
\DeclareMathOperator{\e}{e}
\newcommand{\croit}{\pspicture[shift=*](2,1)\psline{->}(0,0)(2,1)\endpspicture}
\newcommand{\decroit}{\pspicture[shift=*](2,1)\psline{->}(0,1)(2,0)\endpspicture}
%\renewcommand\labelitemi[0]{$\bullet$}
\usepackage{fourier}
\DeclareMathOperator{\card}{card}
\usepackage{siunitx}
\sisetup{
    detect-all,
    output-decimal-marker={,},
    group-minimum-digits = 3,
    group-separator={~},
    number-unit-separator={~},
    inter-unit-product={~}
}

\AtBeginDocument{
  \def\labelitemi{$\bullet$}
}

\makeatother

\usepackage{babel}
\makeatletter
\addto\extrasfrench{%
   \providecommand{\og}{\leavevmode\flqq~}%
   \providecommand{\fg}{\ifdim\lastskip>\z@\unskip\fi~\frqq}%
}

\makeatother
\usepackage{listings}
\providecommand{\exercisename}{Exercice}
\renewcommand{\lstlistingname}{\inputencoding{latin9}Listing}

\begin{document}

\section{Sans algorithmes}

\subsection{Arithmético-géométriques}
\begin{xca}[G1SSMAT02608 ]
 Aujourd'hui les chardons (une plante vivace) ont envahi 300 m\texttwosuperior{}
des champs d’une région. 

Chaque semaine, la surface envahie augmente de 5 \% par le développement
des racines, auquel s’ajoutent 15 m\texttwosuperior{} suite à la dissémination
des graines. 

Pour tout entier naturel $n$, on note $u_{n}$ la surface envahie
par les chardons, en m\texttwosuperior , après $n$ semaines ; on
a donc $u_{0}=300$ m\texttwosuperior . 
\end{xca}
\begin{enumerate}
\item %\mbox{}
\begin{enumerate}
\item Calculer $u_{1}$ et $u_{2}$ .
\item Montrer que la suite $\left(u_{n}\right)$ ainsi définie, n’est ni
arithmétique ni géométrique. 
\end{enumerate}
\end{enumerate}
On admet dans la suite de l’exercice que, pour tout entier naturel
$n$, $u_{n+1}=1,05u_{n}+15$.
\begin{enumerate}[resume]
\item On considère la suite $\left(v_{n}\right)$, définie pour tout entier
naturel $n$, par : $v_{n}=u_{n}+300$.
\begin{enumerate}[resume]
\item Calculer $v_{0}$ , puis montrer que la suite $\left(v_{n}\right)$
est géométrique de raison $q=1,05$. 
\item Pour tout entier naturel $n$, exprimer $v_{n}$ en fonction de $n$,
puis montrer que $u_{n}=600\times1,05^{n}-300$. 
\end{enumerate}
\item Est-il correct d’affirmer que la surface envahie par les chardons
aura doublé au bout de 8 semaines ? Justifier la réponse. 
\end{enumerate}

\begin{xca}[2628]
 La médiathèque d’une petite ville a ouvert ses portes début janvier
2013 et a enregistré 2 500 inscriptions pour l’année 2013. 

Elle estime que, chaque année, 80\% des anciens inscrits renouvellent
leur inscription l’année suivante et qu’il y aura également 400 nouveaux
adhérents. 

Pour tout entier naturel $n$, on peut donc modéliser le nombre d’inscrits
à la médiathèque $n$ années après 2013 par une suite numérique $\left(a_{n}\right)$
définie par : 
\[
a_{0}=2500\text{ et }a_{n+1}=0,8a_{n}+400.
\]
\begin{enumerate}
\item Calculer $a_{1}$ et $a_{2}$ .
\item On pose, pour tout entier naturel $n$, $v_{n}=a_{n}-2000$.
\begin{enumerate}
\item Démontrer que $\left(v_{n}\right)$ est une suite géométrique de raison
$0,8$. Préciser son premier terme.
\item Exprimer, pour tout entier naturel $n$, $v_{n}$ en fonction de $n$. 
\item En déduire que pour tout entier naturel $n$, $a_{n}=500\times0,8^{n}+2000$.
\item Déterminer le plus petit entier naturel $n$ tel que $a_{n}\leqslant2010$.
Interpréter ce résultat dans le contexte de l’exercice. 
\end{enumerate}
\end{enumerate}
\end{xca}

\subsection{Arithmétiques ou Géométriques}

\subsubsection{Géométriques}
\begin{xca}[capital de Lisa, intérêts composés : 2605]

À la naissance de Lisa, sa grand-mère a placé la somme de 5 000 euros
sur un compte et cet argent est resté bloqué pendant 18 ans. Lisa
retrouve dans les papiers de sa grand-mère l’offre de la banque : 

\noindent\fbox{\begin{minipage}[t]{1\columnwidth - 2\fboxsep - 2\fboxrule}%
Offre : 

Intérêts composés au taux annuel constant de 3 \%. 

\textsl{À la fin de chaque année le capital produit 3 \% d’intérêts
qui sont intégrés au capital.} 
%
\end{minipage}}

On considère que l’évolution du capital acquis, en euro, peut être
modélisée par une suite $\left(u_{n}\right)$ dans laquelle, pour
tout entier naturel $n$, $u_{n}$ est le capital acquis, en euro,
n années après la naissance de Lisa. On a ainsi $u_{0}=5000$.
\begin{enumerate}
\item Montrer que $u_{1}=5150$ et $u_{2}=5304,5$.
\item \mbox{}
\begin{enumerate}
\item Pour tout entier naturel $n$, exprimer $u_{n+1}$ en fonction de
$u_{n}$ . En déduire la nature de la suite $\left(u_{n}\right)$
en précisant sa raison et son premier terme.
\item Pour tout entier naturel $n$, exprimer $u_{n}$ en fonction de $n$. 
\end{enumerate}
\item Calculer le capital acquis par Lisa à l’âge de 18 ans. Arrondir au
centième. 
\item Si Lisa n’utilise pas le capital dès ses 18 ans, quel âge aura-t-elle
quand celui-ci dépassera 10 000 euros ?. 
\end{enumerate}
\end{xca}

\paragraph*{Avec sommes}
\begin{xca}[plastiques, sommes de termes : 2609]
 

En 2000, la production mondiale de plastique était de 187 millions
de tonnes. On suppose que depuis 2000, cette production augmente de
3,7 \% chaque année. On modélise la production mondiale de plastique,
en millions de tonnes, produite en l’année $(2000+n)$ par la suite
de terme général $u_{n}$ où $n$ désigne le nombre d’année à partir
de l’an 2000. Ainsi, $u_{0}=187$.
\begin{enumerate}
\item Montrer que la suite $\left(u_{n}\right)$ est une suite géométrique
dont on donnera la raison. 
\item Pour tout $n\in N$, exprimer $u_{n}$ en fonction de $n$.
\item Étudier le sens de variation de la suite $\left(u_{n}\right)$.
\item Selon cette estimation, calculer la production mondiale de plastique
en 2019. Arrondir au million de tonnes. 
\item Des études montrent que 20 \% de la quantité totale de plastique se
retrouve dans les océans, et que 70 \% de ces déchets finissent par
couler. Montrer que la quantité totale, arrondie au million de tonnes,
de déchets flottants sur l'océan dus à la production de plastique
de 2000 à 2019 compris est de 324 millions de tonnes. 
\end{enumerate}
\end{xca}

\subsubsection{Arithmétique}

\paragraph*{hyper trop simples}
\begin{xca}[réseau social : 2604]

En 2019, le nombre d’abonnés à une page de réseau social d’un musicien
était de 6000. On suppose que chaque année, il obtient 750 abonnés
supplémentaires. On désigne par $u_{n}$ le nombre d’abonnés en $2019+n$
pour tout entier naturel $n$.
\begin{enumerate}
\item Calculer le nombre d’abonnés en 2020 et 2021.
\item Exprimer $u_{n+1}$ en fonction de $u_{n}$ .
\item Quelle est la nature de la suite $\left(u_{n}\right)$ ? 
\item En déduire une expression de $u_{n}$ en fonction de $n$.
\item En quelle année le nombre d’abonnés aura triplé par rapport à l’année
2019 ? 
\end{enumerate}
\end{xca}

\paragraph*{avec sommes et second degré}
\begin{xca}[G1SSMAT02599]
 Un artisan commence la pose d’un carrelage dans une grande pièce.
Le carrelage choisi a une forme hexagonale. 

L’artisan pose un premier carreau au centre de la pièce puis procède
en étapes successives de la façon suivante : 
\begin{itemize}
\item à l’étape 1, il entoure le carreau central, à l’aide de 6 carreaux
et obtient une première forme. 
\item à l’étape 2 et aux étapes suivantes, il continue ainsi la pose en
entourant de carreaux la forme précédemment construite. 
\end{itemize}
On note un le nombre de carreaux ajoutés par l’artisan pour faire
la $n-$ième étape ($n\geqslant1$). Ainsi $u_{1}=6$ et $u_{2}=12$. 
\begin{enumerate}
\item Quelle est la valeur de $u_{3}$ ? 
\item On admet que la suite $\left(u_{n}\right)$ est arithmétique de raison
6. Exprimer $u_{n}$ en fonction de $n$. 
\item Combien l’artisan a-t-il ajouté de carreaux pour faire l’étape 5 ?
Combien a-t-il alors posé de carreaux au total lorsqu’il termine l’étape
5 (en comptant le carreau central initial) ?
\item On pose $S_{n}=u_{1}+u_{2}+\cdots+u_{n}$ . Montrer que $S_{n}=6\left(1+2+\cdots+n\right)$
puis que $S_{n}=3n^{2}+3n$. 
\item Si on compte le premier carreau central, le nombre total de carreaux
posés par l’artisan depuis le début, lorsqu’il termine la $n-$ième
étape, est donc $3n^{2}+3n$. \\
À la fin de sa semaine, l’artisan termine la pose du carrelage en
collant son 2977\up{e} carreau. Combien a-t-il fait d’étapes ? 
\end{enumerate}

\end{xca}


\subsection{Comparaisons de suites}

\subsubsection{ventes de voitures, sommes : 2600}

Partie A :

$\left(U_{n}\right)$ est une suite géométrique de premier terme $U_{0}=25000$
et de raison 0,94. $(V_{n})$ est une suite définie par : $V_{n}=50(104+25n)$
pour tout entier naturel $n$. 
\begin{enumerate}
\item Déterminer une forme explicite de la suite $(U_{n})$.
\item Calculer la somme des sept premiers termes de la suite $(U_{n})$. 
\item Comparer les termes $U_{0}$ et $V_{0}$ puis $U_{20}$ et $V_{20}$.
\item Déterminer le plus petit entier naturel $n$ tel que $U_{n}<V_{n}$
. 
\end{enumerate}
Partie B : 

Un concessionnaire de voitures propose des voitures équipées d’un
moteur diesel ou d’un moteur essence. Durant sa première année d’existence
en 1995, il a vendu 25 000 véhicules avec un moteur diesel et 5 200
véhicules avec un moteur essence. Ses ventes de voitures avec un moteur
diesel ont diminué de 6 \% chaque année, alors que ses ventes de voitures
avec un moteur essence ont augmenté de 1 250 unités tous les ans.
En quelle année les ventes de voitures avec un moteur essence ont
elles dépassé les ventes de voitures avec un moteur diesel ? 

\section{Avec algorithmes}

\subsection{Suites (a priori) quelconques}

\subsubsection{Avec variations}
\begin{xca}[G1SSMAT02601 ]
 On considère la suite $\left(u_{n}\right)$ définie pour tout entier
naturel $n$, par $u_{n}=\dfrac{n+2}{n+1}$. 
\begin{enumerate}
\item Calculer $u_{0}$, $u_{1}$, $u_{2}$ puis $u_{99}$ . 
\item %\mbox{}
\begin{enumerate}
\item Exprimer, pour tout entier naturel $n$, $u_{n}-1$ en fonction de
$n$. 
\item Montrer que, pour tout entier naturel $n$, on a : 
\[
u_{n+1}-u_{n}=\dfrac{-1}{\left(n+1\right)\left(n+2\right)}.
\]
\item En déduire le sens de variation de la suite $\left(u_{n}\right)$. 
\end{enumerate}
\item Soit $a$ un nombre réel dans l’intervalle $]1;2]$. Recopier et compléter
sur la copie le programme Python suivant pour qu’il permette de déterminer
le plus petit entier naturel $n$ tel que $u_{n}\leqslant a$ :
\begin{center}
\inputencoding{latin9}\begin{lstlisting}
Def seuil(a) :
    n = 0
    while (n+2) / (n+1) ... a :
        n = ...
    return ...
\end{lstlisting}
\inputencoding{utf8}\par\end{center}

\end{enumerate}
\end{xca}

\begin{xca}[fonction homographique, comparaison : 2603]

On considère les deux suites suivantes : 

la suite $(u_{n})$ définie pour tout entier $n$ par : $u_{n}=\dfrac{8n-4}{n+1}$;

la suite $\left(v_{n}\right)$ définie par $v_{0}=0$ et $v_{n+1}=0,5v_{n}+3,5$
pour tout entier $n$.
\begin{enumerate}
\item Calculer les termes d’indice 3 des suites $\left(u_{n}\right)$ et
$\left(v_{n}\right)$.
\item On s’intéresse aux variations de la suite $\left(u_{n}\right)$. Pour
cela, on considère la fonction $f$ définie sur $[0;+\infty[$ par
: $f\left(x\right)=\dfrac{8x-4}{x+1}$ 
\begin{enumerate}
\item Démontrer que la fonction $f$ est croissante sur $[0;+\infty[$. 
\item En déduire la monotonie de la suite $\left(u_{n}\right)$. 
\end{enumerate}
\item On considère l’affirmation suivante : « pour tout entier $n$, $u_{n}<v_{n}$
». Camille pense que cette affirmation est vraie alors que Dominique
pense le contraire. Pour les départager, on réalise le programme suivant
écrit en langage Python :
\begin{lyxcode}
def~algo(~)~:

~~~~n~=~0

~~~~u~=~-~4

~~~~v~=~0

~~~~while~u~<~v~:

~~~~~~~~n~=~n+1

~~~~~~~~u~=~(8{*}n~-~4)/(n+1)

~~~~~~~~v~=~0.5{*}v~+~3.5

~~~~return(n)~
\end{lyxcode}
Le programme renvoie la valeur 11. Qui de Camille ou Dominique a raison
? Expliquer. 
\end{enumerate}
\end{xca}

\subsection{Arithmético-géométriques}
\begin{xca}[G1SSMAT02651 ]
 La bibliothèque municipale étant devenue trop petite, une commune
a décidé d’ouvrir une médiathèque qui pourra contenir 100 000 ouvrages
au total. Pour l’ouverture prévue le 1er janvier 2020, la médiathèque
dispose du stock de 35 000 ouvrages de l’ancienne bibliothèque, augmenté
de 7 000 ouvrages supplémentaires neufs offerts par la commune. 

Partie A 

Chaque année, le bibliothécaire est chargée de supprimer 5\% des ouvrages,
trop vieux ou abîmés, et d’acheter 6 000 ouvrages neufs. On appelle
$u_{n}$ le nombre, en milliers, d’ouvrages disponibles le 1er janvier
de l’année $(2020+n)$. 

On donne $u_{0}=42$. 
\begin{enumerate}
\item Justifier que, pour tout entier naturel $n$, on a $u_{n+1}=u_{n}\times0,95+6$. 
\item On propose ci-dessous un programme en langage Python. Expliquer ce
que permet de déterminer ce programme. 

\inputencoding{latin9}\begin{lstlisting}[language=Python]
def suite(n) :
    u=42
    for i in range(n) :
        u=0.95*u+6 
    return u 
\end{lstlisting}
\inputencoding{utf8}
\end{enumerate}
Partie B 

La commune doit finalement revoir ses dépenses à la baisse, elle ne
pourra financer que 4 000 nouveaux ouvrages par an au lieu des 6 000
prévus. 

On appelle $v_{n}$ le nombre, en milliers, d’ouvrages disponibles
le 1er janvier de l’année $(2020+n)$. 
\begin{enumerate}
\item On admet que $v_{n+1}=v_{n}\times0,95+4$ pour tout entier naturel
$n\geqslant0$ avec $v_{0}=42$. \\
On considère la suite $\left(w_{n}\right)$ définie, pour tout entier
naturel $n$, par $w_{n}=v_{n}-80$. 
\begin{enumerate}
\item Montrer que $\left(w_{n}\right)$ est une suite géométrique de raison
$q=0,95$ et préciser son premier terme $w_{0}$ .
\item En déduire l’expression de $w_{n}$ puis de $v_{n}$ en fonction de
$n$.
\end{enumerate}
\item On donne ci-dessous un programme en langage Python. L’appel à la fonction
objet(70) renvoie 27. Interpréter ce résultat dans le contexte de
l’exercice.

\inputencoding{latin9}\begin{lstlisting}[language=Python]
def objet(A) : 
    v=42 
    n=0 
    while v<A : 
        v=0.95*v+4 
        n=n+1 
    return n 
\end{lstlisting}
\inputencoding{utf8}
\end{enumerate}
\end{xca}


\subsubsection{avec somme de termes}
\begin{xca}[G1SSMAT02602 ]
 Soit la suite $\left(u_{n}\right)$ de premier terme $u_{0}=400$
vérifiant la relation, pour tout entier naturel $n$, 
\[
u_{n+1}=0,9u_{n}+60.
\]
Soit la suite géométrique $\left(v_{n}\right)$ de premier terme $v_{0}=-200$
et de raison $0,9$. 
\end{xca}
\begin{enumerate}
\item Calculer $u_{2}$ et $v_{2}$ . 
\item Calculer la somme des 20 premiers termes de la suite $\left(v_{n}\right)$. 
\item La suite $\left(u_{n}\right)$ est-elle arithmétique ? Est-elle géométrique
? 
\item Recopier et compléter la fonction Suite suivante écrite en Python
qui permet de calculer la somme $S$ des 20 premiers termes de la
suite $\left(u_{n}\right)$. 

\inputencoding{latin9}\begin{lstlisting}[language=Python]
def Suite( ) :
    U = 400
    S = 0
    for i in range(20) :
        S = ..........
        U = ..........
    return(...)
\end{lstlisting}
\inputencoding{utf8}
\item On admet que $u_{n}=v_{n}+600$. En déduire $u_{20}$. 
\end{enumerate}

\subsection{Arithmétiques ou Géométriques}

\subsubsection{Géométriques}
\begin{xca}[G1SSMAT02597 ]
 Lors du lancement d’un hebdomadaire, 1 200 exemplaires ont été vendus. 

Une étude de marché prévoit une progression des ventes de 2 \% chaque
semaine. 

On modélise le nombre d’hebdomadaires vendus par une suite $\left(u_{n}\right)$
où $u_{n}$ représente le nombre de journaux vendus durant la $n$-ième
semaine après le début de l’opération. 

On a donc $u_{0}=\num{1200}$.
\end{xca}
\begin{enumerate}
\item Calculer le nombre $u_{2}$ . Interpréter ce résultat dans le contexte
de l’exercice. 
\item Écrire, pour tout entier naturel $n$, l’expression de $u_{n}$ en
fonction de $n$. 
\item Voici un programme rédigé en langage Python : 
\begin{center}
\inputencoding{latin9}\begin{lstlisting}[language=Python]
def suite( ) :
    u = 1200
    S = 1200
    n = 0
    while a < 30000 :
        n = n+1
        u = u*1.02
        S=S+u
    return(n)
\end{lstlisting}
\inputencoding{utf8}\par\end{center}

Le programme retourne la valeur 30 . Interpréter ce résultat dans
le contexte de l’exercice. 
\item Déterminer le nombre total d’hebdomadaires vendus au bout d’un an. 
\end{enumerate}

\begin{xca}[voiture : 2607]
 

En 2002, Camille a acheté une voiture, son prix était alors de 10
500 €. La valeur de cette voiture a baissé de 14 \% par an. 
\begin{enumerate}
\item La valeur de cette voiture est modélisée par une suite. On note $P_{n}$
la valeur de la voiture en l’année $2002+n$. On a donc $P_{0}=10500$. 
\begin{enumerate}
\item Déterminer la nature de la suite $\left(P_{n}\right)$. 
\item Quelle était la valeur de cette voiture en 2010 ? 
\end{enumerate}
\item Camille aimerait savoir à partir de quelle année la valeur de sa voiture
est inférieure à 1500 €. Pour l’aider, on réalise le programme Python
incomplet ci-dessous. 
\begin{enumerate}
\item Recopier et compléter sur votre copie les deux parties en pointillé
du programme ci-dessous : 
\begin{lyxcode}
def~algo(~)~:

~~~~P=10500

~~~~n=2002

~~~~while~P~..................~:

~~~~~~~~P=.................~

~~~~~~~~n=n+1~

~~~~return(n)~
\end{lyxcode}
\item Donner la valeur renvoyée par ce programme. 
\end{enumerate}
\end{enumerate}
\end{xca}

\subsubsection{Géométriques avec somme}
\begin{xca}[G1SSMAT02606 ]
 Un service de vidéos à la demande réfléchit au lancement d’une nouvelle
série mise en ligne chaque semaine et qui aurait comme sujet le quotidien
de jeunes gens favorisés. 

Le nombre de visionnages estimé la première semaine est de 120 000.
Ce nombre augmenterait ensuite de 2\% chaque semaine. 

Les dirigeants souhaiteraient obtenir au moins 400 000 visionnages
par semaine. 

On modélise cette situation par une suite $\left(u_{n}\right)$ où
$u_{n}$ représente le nombre de visionnages $n$ semaines après le
début de la diffusion. On a donc $u_{0}=\num{120000}$. 
\end{xca}
\begin{enumerate}
\item Calculer le nombre $u_{1}$ de visionnages une semaine après le début
de la diffusion.
\item Justifier que pour tout entier naturel $n$ : $u_{n}=\num{12000}\times1,02^{n}$
.
\item À partir de combien de semaines le nombre de visionnages hebdomadaire
sera-t-il supérieur à 150 000 ? 
\item Voici un algorithme écrit en langage Python : 

\inputencoding{latin9}\begin{lstlisting}[language=Python]
def seuil():
    u=120000
    n=0
    while u<400000:
        n=n+1
        u=1.02*u
    return n
\end{lstlisting}
\inputencoding{utf8}
Déterminer la valeur affichée par cet algorithme et interpréter le
résultat précédent dans le contexte de l’exercice.
\item On pose pour tout entier naturel $n$ : $S_{n}=u_{0}+\cdots+u_{n}$
. Montrer que l’on a : 
\[
S_{n}=\num{6000000}\times\left(1,02^{n+1}-1\right)
\]
 Puis en déduire le nombre total de visionnages au bout de 52 semaines
(arrondir à l’unité). 
\end{enumerate}

\begin{xca}[G1SSMAT02598]
 

Partie A 

Soit $\left(u_{n}\right)$ une suite géométrique de raison 2 de premier
terme u\_0 =0,2 . 
\begin{enumerate}
\item Calculer $u_{18}$ puis $u_{50}$ . 
\item Calculer $u_{0}+u_{1}+\cdots+u_{18}$ , c’est-à-dire la somme des
19 premiers termes de la suite $\left(u_{n}\right)$.
\item Recopier et compléter les trois parties en pointillé de l’algorithme
suivant permettant de déterminer le plus petit entier $n$ tel que
la somme des $n+1$ premiers termes de la suite $u$ dépasse 100 000. 
\begin{lyxcode}
U~←~0,2

S~←~0,2

N~←0

Tant~que~............................

~~~~U~←..........

~~~~S~←~..........

~~~~N~←N+1

Fin~tant~que

Afficher~N
\end{lyxcode}
\end{enumerate}
Partie B 

Claude a donné 20 centimes d’euros (soit 0,20 €) à son petit-enfant
Camille pour sa naissance. Ensuite, Claude a doublé le montant offert
d’une année sur l’autre pour chaque anniversaire jusqu’aux 18 ans
de Camille. 

La somme totale versée par Claude à Camille permet-elle de payer un
appartement à Angers d’une valeur de 100 000 € ? 
\end{xca}

\end{document}
